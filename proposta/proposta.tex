%        File: planoic.tex
%     Created: Fri Apr 18 01:00 PM 2014 B
% Last Change: Fri Apr 18 01:00 PM 2014 B
%
\documentclass[12pt, a4paper, oneside]{article}

\usepackage{amsfonts}
\usepackage[]{amsmath}
\usepackage{amssymb}
\usepackage[brazil]{babel}
\usepackage[pt-BR]{datetime2}
\usepackage[left=20mm,right=20mm,top=25mm,bottom=25mm]{geometry}
\usepackage[]{graphicx}
\usepackage[utf8]{inputenc}
\usepackage{multirow}
\usepackage{paralist}
\usepackage{setspace}
\usepackage[compact]{titlesec}
\usepackage[]{url}
\usepackage{xspace}
\usepackage{mathpazo}
\usepackage{txfonts}

\titleformat{\section}
  {\Large\bfseries}{\thesection.}{.5em}{}[\hrule\bigskip]
\titleformat{\subsection}
  {\bfseries}{\thesection.}{.5em}{}
%\renewcommand{\baselinestretch}{1.5} 
%\renewcommand{\rmdefault}{phv} % Arial
%\renewcommand{\sfdefault}{phv} % Arial
\setlength{\parskip}{6pt}

\newcommand{\A}{\ensuremath{\mathtt{A}}\xspace}
\newcommand{\C}{\ensuremath{\mathtt{C}}\xspace}
\newcommand{\G}{\ensuremath{\mathtt{G}}\xspace}
\newcommand{\T}{\ensuremath{\mathtt{T}}\xspace}

\newcommand{\str}[1]{\ensuremath{\mathtt{#1}}\xspace}
\newcommand{\strset}[1]{\ensuremath{\mathcal{#1}}\xspace}
\newcommand{\ssS}{\strset{S}}
\newcommand{\seq}[1]{\ensuremath{\mathtt{#1}}\xspace}

\newcommand{\rank}{\ensuremath{\mathrm{rank}}\xspace}
\newcommand{\select}{\ensuremath{\mathrm{select}}\xspace}
\newcommand{\BWT}{\ensuremath{\mathrm{BWT}}\xspace}

\newcommand{\X}{\ensuremath{\medbullet}\xspace}
\newcommand{\x}{\ensuremath{\medcirc}\xspace}




\newcommand{\thetitle}{Contrução online de wavelet tree baseada no código de Huffman}
\newcommand{\workarea}{Teoria da Computação}
\newcommand{\major}{Engenharia da Computação}
\newcommand{\studenttitle}{Aluno}
\newcommand{\student}{Ícaro Julião Monteiro Guerra}
\newcommand{\studentemail}{ijmg@cin.ufpe.br}
\newcommand{\advisertitle}{Orientador}
\newcommand{\adviser}{Paulo Fonseca}
\newcommand{\adviseremail}{paguso@cin.ufpe.br}


%\bibliography{projeto}


\begin{document}
%\onehalfspacing

%\maketitle

{\Large
\thispagestyle{empty}
\begin{center}
\begin{tabular}{l p{10cm}}
	\includegraphics[width=3cm]{cin-logo.png} &
	\vspace{-25mm}
	Universidade Federal de Pernambuco\newline
	Centro de Informática\newline
	Graduação em \major
\end{tabular}

\vfill

{\huge \bfseries \thetitle}
\\
\medskip
{\bfseries\itshape Proposta de Trabalho de Graduação}

\vfill

\bigskip

	\begin{tabular}{r p{95mm}}
	\textbf{\studenttitle: } & \student \newline(\texttt{\studentemail}) \\ 
\textbf{\advisertitle: } & \adviser \newline(\texttt{\adviseremail})
\\
	\textbf{Área: } & \workarea 
\end{tabular}

	\vspace{3cm}
\DTMlangsetup{showdayofmonth=false}
Recife, \today 
\DTMlangsetup{showdayofmonth=true}
\end{center}
}

\normalfont

\clearpage
\setcounter{page}{1}
\section*{Resumo}

A wavelet tree é uma estrutura de dados que permite guardar cadeias de caracteres em um espaço comprimido muito utilizada para a indexação de textos. Esse projeto tem como objetivo propor um algoritmo para construção \emph{online} de uma wavelet tree utilizando de base a estrutura da codificação do texto a partir do código de Huffman.

\clearpage
\section*{Cronograma de atividades}


\begin{center}
	\begin{tabular}{| l || c | c | c | c | c | c | c | c | c | c | c | c | c | c | c |  c | c | c | c | c |}
		\hline
		& \multicolumn{2}{| c |}{Dez} & \multicolumn{4}{| c |}{Jan} & \multicolumn{4}{| c |}{Fev} & \multicolumn{4}{| c |}{Mar} & \multicolumn{3}{| c |}{Abr} \\\hline\hline
		Preparação da proposta & \X & \X & & & & & & & & & & & & & & & \\\hline 
		Revisão bibliográfica$^\dagger$ & \X & \X & \x & \x & \x & \x & \x & \x & & & & & & & & &\\\hline 
		Implementação das estruturas & & & \X & \X & \X & \X & \X & \X & \X & \X & \X & & & & &  &\\\hline 
		Realização dos testes$^\ddagger$ & & & & \x & \x & \x & \x & \x & \x  & \X & \X & \X & & & & & \\\hline 
		Redação e revisão da monografia & & & & & & & & & & & \X & \X & \X & \X & & & \\\hline 
		Preparação da apresentação & & & & & & & & & & & & & & & \X & \X & \X \\\hline 
\hline
	\end{tabular}
\begin{minipage}{0.6\linewidth}
\noindent($\dagger$) \X = levantamento inicial, \x= aprofundamento\newline
\noindent($\ddagger$) \X= experimentos formais , \x = testes \textit{ad hoc} unitários\newline
\end{minipage}

\end{center}


\clearpage
\section*{Possíveis Avaliadores}

\begin{enumerate}
\item Prof. Nivan Ferreira
\item Prof. Gustavo Carvalho
\end{enumerate}


\clearpage
\section*{Assinaturas}

\vfill
\begin{center}
	Recife, \today 

	\vspace{3cm}
	\rule{10cm}{.5pt}\\
	\textbf{\studenttitle:} \student\\

	\vspace{3cm}
	\rule{10cm}{.5pt}\\
	\textbf{\advisertitle:} \adviser\\
\end{center}
\vfill

\end{document}

